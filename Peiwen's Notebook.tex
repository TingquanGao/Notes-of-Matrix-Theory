\documentclass{ctexart}

% 调整页面布局
\usepackage{geometry}
\geometry{a4paper, centering, scale=0.8}

% 定义圈1、圈2
\usepackage{tikz}
\newcommand*{\circled}[1]{\lower.7ex\hbox{\tikz\draw (0pt, 0pt)%
    circle (.5em) node {\makebox[1em][c]{\small #1}};}}

\usepackage{amsmath}
% \usepackage{amsthm}
\usepackage{ntheorem}

% because therefore
\usepackage{amssymb}


\title{\heiti 培文笔记}
\author{\kaishu 陈培文}
\date{\today}

% 数学专有定义
\newtheorem*{example}{\hspace{2em}例} % [section]
\newtheorem{algorithm}{\hspace{2em}算法}[section]
\newtheorem{definition}{\hspace{2em}定义}[section]
\newtheorem{axiom}{\hspace{2em}公理}[section]
\newtheorem{theorem}{\hspace{2em}定理}[section]
\newtheorem{lemma}{\hspace{2em}引理}[section]
\newtheorem{proposition}{\hspace{2em}命题}[section]
\newtheorem{corollary}{\hspace{2em}推论} % [section]
\newtheorem{remark}{\hspace{2em}注解}[section]
\newtheorem{property}{\hspace{2em}性质}[section]
\newtheorem{condition}{\hspace{2em}条件}[section]
\newtheorem{conclusion}{\hspace{2em}结论}[section]
\newtheorem{assumption}{\hspace{2em}假设}[section]
\newtheorem*{proof}{\hspace{2em}证明} % [section]
\newtheorem*{solutiontitle}{\hspace{2em}解答}

\begin{document}
    \maketitle
    \thispagestyle{empty}
    \newpage
    \tableofcontents    %生成目录
    \thispagestyle{empty}   %目录页不显示页码
    \newpage
    \setcounter{page}{1}    %从下面开始编页码
    
    % 页码使用罗马数字
    % \pagenumbering{Roman}
    
    % 第一章
    \section{线性空间的定义}
\subsection{线性空间的定义}
用F表示实数全体($R$)或复数全体($C$)

\begin{definition}
    设$V$是非空集合,$F$是实数($R$)或复数($C$)域
\end{definition}

在$V$及$F$上定义了两种运算:
\begin{definition}[加法]
    对$\forall\alpha,\beta\in V$,在$V$中有唯一的元素与之对应,记这个元素为$\alpha+\beta$,称为$\alpha,\beta$的和。
\end{definition}
\begin{definition}[数乘]
    对$\forall \alpha \in V, k \in F$,在$V$中有唯一对元素与之对应,记这个元素为$k\cdot\alpha$,称为$k$与$\alpha$的积。
\end{definition}    

如果满足下述公理,则称$V$是数域$F$上的线性空间,$V$中的元素称为是向量。
\begin{enumerate}
    \item 加法运算
    \begin{axiom}[交换律]
        对$\forall \alpha,\beta \in V, \alpha+\beta=\beta+\alpha$
    \end{axiom}

    \begin{axiom}[结合律]
        对$\forall \alpha, \beta, \gamma \in V, (\alpha+\beta)+\gamma=\alpha+(\beta+\gamma)$
    \end{axiom}

    % 零元素此处未完成
    \begin{axiom}[零元素]
        $\exists \theta \in V$,使得$\forall \alpha \in V, \alpha + \theta = \alpha$
    \end{axiom}

    \begin{axiom}[负元素]
        对$\forall \alpha \in V, \exists \beta \in V$,使$\alpha+\beta=\theta$
    \end{axiom}

    \item 数乘运算
    \begin{axiom}
        对$\forall \alpha \in V, 1 \cdot \alpha = \alpha$
    \end{axiom}

    \begin{axiom}
        对$\forall \alpha \in V, k,l \in F, k\cdot(l \cdot f) = (k \cdot l) \cdot f$
    \end{axiom}

    \item 数乘和加法运算
    \begin{axiom}
        对$\forall \alpha \in V, k,l \in F, (k+l)\cdot \alpha = k \alpha + l \alpha$
    \end{axiom}
    \begin{axiom}
        对$\forall \alpha, \beta \in V, k \in F, k(\alpha + \beta) = k \alpha + k \beta$
    \end{axiom}
\end{enumerate}

\begin{example}
    \begin{enumerate}
        \item $n$维向量,$V = F^n$
        \item $n \times n$维矩阵全体,$V = F^{n \times n}$
        \item 系数在$F$中关于$x$的多项式全体,$V = F[x]$
        \item $V = F_n[x]$ % 此处尚未完成
        \item $V = C, F = R$
        \item $V = C, F = C$
        \item 不构成线性空间,$V = R, F = C$
        \item 通常运算,$R^+$:正实数全体,不构成线性空间,$V = R^+, F = R$
        \item $V = R^+, F = R$
    \end{enumerate}
\end{example}

定义新的运算:
\begin{definition}[$\bigoplus$]
    对$\alpha, \beta \in V, \alpha \bigoplus \beta = \alpha \cdot \beta$
\end{definition}

\begin{definition}[$\circ$]
    对$\alpha \in V, k \in F, k \circ \alpha = \alpha^k$
\end{definition}

验证:
\begin{itemize}
    \item[\circled{1}] $\alpha, \beta \in V, \alpha \bigoplus \beta = \alpha \beta, \beta \bigoplus \alpha = \beta \alpha, \alpha \bigoplus \beta = \beta \bigoplus \alpha$
    \item[\circled{3}] $\theta in V, \forall \alpha \in V, \alpha \bigoplus \theta = \alpha, \theta = 1$
    \item[\circled{4}] $\alpha \in V, \alpha \bigoplus \beta = \theta = 1, \beta = \alpha^{-1}$
    \item[\circled{7}] $k, l \in F, \alpha \in V, (k + l) \circ \alpha \overset{?}{=} k \circ \alpha \bigoplus l \circ \alpha$   
\end{itemize}

\begin{property}[线性空间的性质]
    假设$V$是数域$F$上的线性空间,则:
    \begin{enumerate}
        \item $V$中的零向量是唯一的
        \begin{proof}
            若有$2$个零元素$\theta_1, \theta_2$,则:
            \begin{equation}
                \theta_1 = \theta_1 + \theta_2 = \theta_2 + \theta_1 = \theta_2,
            \end{equation}
            故$\theta_1 = \theta_2$,零元素唯一。
        \end{proof}
        
        \item 对$\forall \alpha \in V, \alpha$的负元素是唯一的,记为$-\alpha$
        \begin{proof}
            设$\beta_1, \beta_2$均为$\alpha$的负元素
            \begin{equation}
                \beta_1 = \beta_1 + \theta = \beta_1 + (\alpha + \beta_2) = (\beta_1 + \alpha) + \beta_2 = (\alpha + \beta_1) + \beta_2 = \theta + \beta_2 = \beta_2 + \theta = \beta_2
            \end{equation}
        \end{proof}

        \item 加法消去率:若$\alpha + \beta  = \alpha + \gamma$,则$\beta = \gamma$
        \item 对$\forall \alpha, \beta \in V$,向量方程$\alpha + x = \beta$有唯一解,$x = \alpha + (-\beta)$,记$x = \alpha - \beta$
        \item $(-k) \cdot \alpha = -(k \alpha)$,特别地,$(-1) \alpha = \alpha$
        \item $k \alpha \Leftrightarrow k = 0$或$\alpha =0$
    \end{enumerate}

    \subsection{基、维数和坐标}
    在线性空间中,可以定义线性组合、线性表、线性相关、线性无关、向量组的极大线性无关组、秩等概念,如:
    \begin{definition}
        设$\alpha_1, \alpha_2, \cdots, \alpha_s \in V$,若存在不全为$0$的数$k_1, k_2, \cdots, k_s$,使得$k_1 \alpha_1 + k_2 \alpha_2 + \cdots + k_s \alpha_s = 0$,则称向量组$\alpha_1, \alpha_2, \cdots, \alpha_s$线性相关,否则,称$\alpha_1, \alpha_2, \cdots, \alpha_s$是线性无关的。
    \end{definition}
    
    一些重要结论:
    \begin{enumerate}
        \item 
        \begin{conclusion}
            若$s \geq 2$,则$\alpha_1, \alpha_2, \cdots, \alpha_s$线性相关$\Leftrightarrow$ $\exists j$,使$\alpha_j$可由其余$s - 1$个向量线性表示。
            \begin{proof}
                存在不全为$0$的数,$k_1, k_2, \cdots, k_s$,使$k_1 \alpha_1 + k_2 \alpha_2 + \cdots + k_s \alpha_s = 0$
                不妨设$k_1 ≠ 0 \Rightarrow \alpha_1 = -\frac{k_1}{k_1} \alpha_2 - \cdots - \frac{k_s}{k_1} \alpha_s$
            \end{proof}
        \end{conclusion}

        \item
        \begin{conclusion}
            若$\alpha_1, \alpha_2, \cdots, \alpha_s$线性无关,但$\beta, \alpha_1, \alpha_2, \cdots, \alpha_s$线性相关,则$\beta$可由$\alpha_1, \alpha_2, \cdots, \alpha_s$线性表示,而且,线性表示的方法是唯一的。
        \end{conclusion}

        \item
        \begin{conclusion}
            若$t>s, \beta_1, \beta_2, \cdots, \beta_s$可由$\alpha_1, \alpha_2, \cdots, \alpha_s$线性表示,则$\beta_1, \beta_2, \cdots, \beta_t$线性相关
            \begin{proof}
                % 存疑
                极大线性无关组:$\alpha_1, \alpha_2, \cdots, \alpha_s$中的极大线性无关组$\alpha_{i_1}, \alpha_{i_2}, \cdots, \alpha_{i_r}$满足$2$个条件:
                \begin{itemize}
                    \item[\circled{1}] $\alpha_{i_1}, \alpha_{i_2}, \cdots, \alpha_{i_r}$线性无关
                    \item[\circled{2}] $\alpha_1, \alpha_2, \alpha_s$中的每一个向量均可由$\alpha_{i_1}, \alpha_{i_2}, \cdots, \alpha_{i_r}$线性表示  
                \end{itemize}
                \begin{corollary}
                    若$\beta_1, \beta_2, \cdots, \beta_t$可由$\alpha_1, \alpha_2, \cdots, \alpha_s$线性表示,且$\beta_1, \beta_2, \cdots, \beta_t$线性无关,则$t \leq s$
                \end{corollary}
                \begin{corollary}
                    若$\beta_1, \beta_2, \cdots, \beta_t$与$\alpha_1, \alpha_2, \cdots, \alpha_s$等价,且均线性无关,则$s = t$
                \end{corollary}
            \end{proof}
        \end{conclusion}
    \end{enumerate}

    \begin{example}
        \begin{enumerate}
            \item 在$F^{2 \times 2}$中,$
            E_{11} = {\left(
                \begin{array}{ccc}
                    1 & 0 \\
                    0 & 0
                \end{array}\right)},
            E_{12} = \left(
                \begin{array}{ccc}
                    0 & 1 \\
                    0 & 0
                \end{array}\right),
            E_{21} = \left(
                \begin{array}{ccc}
                    0 & 0 \\
                    1 & 0
                \end{array}\right),
            E_{22} = \left(
                \begin{array}{ccc}
                    0 & 0 \\
                    0 & 1
                \end{array}\right)$
                \begin{solutiontitle}
                    设 $\k_1 E_{11} + k_2 E_{12} + k_3 E_{21} + k_4 E_{22} = 0$
                    $\Rightarrow 
                    \left(\begin{array}{ccc}
                        k_1 & k_2 \\
                        k_3 & k_4
                    \end{array}\right) = 0$
                    $\therefore k_1 = k_2 = k_3 = k_4 = 0$
                \end{solutiontitle}

                \item 在$F_3[x]$中,$\alpha_1 = 2 + x + 3x^2, \alpha_2 = 1 + 3x - x^2, \alpha_3 = 3 + 4x + 2x^2$
                $k_1 \alpha_1 + k_2 \alpha_2 + k_3 \alpha_3 = \theta$,是否有不全为$0$的$k_1, k_2, k_3$使上式成立
                \begin{solutiontitle}
                    $\Rightarrow (2k_1 + k_2 + 3k_3) + (k_1 + 3k_2 + 4k_3)x + (3k_1 - k_2 + 2k_3)x^2 = 0 \\
                    \Rightarrow \left\{
                        \begin{array}{ccc}
                            2k_1 + k_2 + 3k_3 = 0 \\
                            k_1 + 3k_2 + 4k_3 = 0 \\
                            3k_1 - k_2 + 2k_3 = 0
                        \end{array}
                    \right.$
                    $\Rightarrow \left\{
                        \begin{array}{ccc}
                            k_1 = 1 \\
                            k_2 = 1 \\
                            k_3 = -1
                        \end{array}
                    \right.$
                    $\Rightarrow \alpha_1, \alpha_2, \alpha_3$线性相关
                \end{solutiontitle}
                
                \item $V = C, F = R, \alpha_1 = 1, \alpha_2 = i = -\sqrt{-1}$
                是否存在$a, b \in F = R, a\alpha_1 + b\alpha_2 = 0$\\
                $\Rightarrow a + b_i = 0 \Rightarrow a = b = 0 \Rightarrow \alpha_1$与$\alpha_2$线性无关

                \item $V = C, F = R, \alpha_1 = 1, \alpha_2 = i = -\sqrt{-1}$
                是否存在$a, b \in F = C$,使得$a\alpha_1 + b\alpha_2 = 0$\\
                若$a = i = \sqrt{-1}, b = -1$,则$a\alpha_1 + b\alpha_2 = 0 \Rightarrow \alpha_1, \alpha_2$线性相关\\
                \emph{小例子}:$V = R^+, F = R$,两种运算$\bigoplus, \circ, \alpha = 2, \beta = 3 \in V$\\
                $k \circ \alpha \bigoplus l \circ \beta = \theta \Rightarrow \alpha^k \cdot \beta^l = 1 \Rightarrow 2^k + 3^l =1 \Rightarrow k = 1$,则$l = -\log_32$
        \end{enumerate}
    \end{example}

    \begin{definition}{基,维数}\\
        若$\alpha_1, \alpha_2, \cdots, \alpha_n \in V$满足条件:
        \begin{enumerate}
            \item $\alpha_1, \alpha_2, \cdots, \alpha_n$线性无关
            \item $\forall \eta \in V$均可由$\alpha_1, \alpha_2, \cdots, \alpha_n$线性表示
        \end{enumerate}
        称$n$是$V$的维数,记为维($V$)或$\dim V$
    \end{definition}
    
    \emph{注}:
    \begin{proposition}
        若$\dim V = n$,则$V$中任意$n+1$个向量线性相关\\
        线性空间的基不一定存在。\\
        \emph{例如}:零空间$V = {\theta}, \dim{\theta} = 0$\\
        $V = F[x], \dim F[x] = \infty$
    \end{proposition}

    \begin{example}
        \begin{enumerate}
            \item $V = F^n$
            \begin{solutiontitle}
                $e_1 = (1, 0, \cdots, 0), e_2 = (0, 1, \cdots, 0), \cdots, e_n = (0, 0, \cdots, 1)$\\
                $\eta = (x_1, x_2, \cdots, x_n) = x_1e_1 + x_2e_2 + \cdots + x_ne_n$\\
                其中$e_1, e_2, \cdots, e_n$为$V$的自然基,$\dim F^n = n$
            \end{solutiontitle}

            \item $V = F^{2 \times 2}$
            \begin{solutiontitle}
                $E_{11} = \left(
                    \begin{array}{ccc}
                        1 & 0 \\
                        0 & 0
                    \end{array}\right),
                E_{12} = \left(
                    \begin{array}{ccc}
                        0 & 1 \\
                        0 & 0
                    \end{array}\right),
                E_{21} = \left(
                    \begin{array}{ccc}
                        0 & 0 \\
                        1 & 0
                    \end{array}\right),
                E_{22} = \left(
                    \begin{array}{ccc}
                        0 & 0 \\
                        0 & 1
                    \end{array}\right)$\\
                $\dim F^{2 \times 2} = 4$\\
                $F^{3 \times n}, F,$基:矩阵单位$\{E_{ij}\}$,维数:$\dim F^{s \times n} = s \times n$
            \end{solutiontitle}

            \item $V = F_n[x]$
            \begin{solutiontitle}
                基:$1, x, x^2, \cdots, x^{n-1}, \dim F_n[x] = n$
            \end{solutiontitle}

            \item $V = C, F = R$(基:$1, \sqrt{-1}$)
            \item $V = C, F = C$(基:$1, \dim V = 1$)
            \item $V = R^+, F = R, \bigoplus, \circ, \theta = 1$(基:$\alpha \neq 1即可, \dim V = 1$)
        \end{enumerate}
    \end{example}
    
    \begin{example}
        证明:在$F_3[x]$中,下述三个向量构成一组基:
        $f_1(x) = 1 + 2x + 3x^2, f_2(x) = 3 + x - x^2, f_3(x) = 2 - x + x^2$
        \begin{itemize}
            \item[方法一:]
            \begin{itemize}
                \item[\circled{1}] 说明$f_1(x), f_2(x), f_3(x)$线性无关
                \item[\circled{2}] $\forall p(x)$均可由$f_2(x), f_3(x)$线性表示
            \end{itemize} 
             设$k_1 f_1 + k_2f_2 + k_3f_3 = \theta \Rightarrow k_1 = k_2 = k_3 = 0$\\
             $p(x) = a + bx + cx^2$,寻找使$p(x) = k_1f_1 + k_2f_2 + k_3f_3$
            \item[方法二:]
            $\dim F_3[x] = 3$,且$f_1, f_2, f_3$线性无关\\
            由上述定理可知,$f_1, f_2, f_3$为一组基  
        \end{itemize}
    \end{example}

    \begin{definition}[坐标]
        设$\alpha_1, \alpha_2, \cdots, \alpha_n$是$V$的一组基,$\beta \in V$且$\beta = x_1\alpha_1 + x_2\alpha_2 + \cdots + x_n\alpha_n$,则称$x_1, x_2, \cdots, x_n$是$\beta$在基$a\alpha_1, \alpha_2, \cdots, \alpha_n$下的坐标,或$(x_1, x_2, \cdots, x_n)^T$是$\beta$在基$\alpha_1, \alpha_2, \cdots, \alpha_n$下的坐标(列向量)。
    \end{definition}

    \begin{example}
        \begin{enumerate}
            \item $F^n$中,$\eta = (x_1, x_2, \cdots, x_n)$在基$e_1 = (1, 0, \cdots, 0), e_2 = (0, 1, \cdots, 0), \cdots, e_n = (0, 0, \cdots, 1)$下的坐标\\
            $\eta = x_1e_1 + x_2e_2 + \cdots + x_ne_n \Rightarrow$坐标$(x_1, x_2, \cdots, x_n)^T$

            \item 在$F^{2 \times 2}$中,
            $A = \left(\begin{array}{ccc}
                a & b \\
                c & d
            \end{array}\right)$在基
            $E_{11} = \left(\begin{array}{ccc}
                1 & 0 \\
                0 & 0
            \end{array}\right),
            E_{12} = \left(\begin{array}{ccc}
                0 & 1\\
                0 & 0
            \end{array}\right),
            E_{21} = \left(\begin{array}{ccc}
                0 & 0\\
                1 & 0
            \end{array}\right),
            E_22 = \left(\begin{array}{ccc}
                0 & 0\\
                0 & 1
            \end{array}\right)$下的坐标\\
            $A = aE_{11} + bE_{12} + cE_{21} + dE_{22} \Rightarrow$坐标
            $\left(\begin{array}{ccc}
                a\\
                b\\
                c\\
                d
            \end{array}\right)$
        \end{enumerate}
    \end{example}

    \emph{注}:
    \begin{enumerate}
        \item 线性空间的基是有序的\\
        $F^n, \eta = (x_1, x_2, \cdots, x_n)$在$e_1, e_2, \cdots, e_n$下坐标
        $\left(\begin{array}{ccc}
            x_1 \\
            x_2 \\
            \vdots \\
            x_n
        \end{array}\right)$,
        在$e_n, e_{n-1}, \cdots, e_1$下坐标
        $\left(\begin{array}{ccc}
            x_n \\
            x_{x_1} \\
            \vdots \\
            x_1
        \end{array}\right)$

        \item 基的几何意义
        \begin{theorem}
            假设$\eta, \eta_i \in V$在基$\alpha_1, \alpha_2, \cdots, \alpha_n$下的坐标分别是$X$及$X_i, i = 1, 2, \cdots, s$,则:
            \begin{enumerate}
                \item $\eta = \theta  \Leftrightarrow X = \theta$
                \item $\eta = k_1\eta_1 + k_2\eta_2 + \cdots, k_s\eta_s \Leftrightarrow X = k_1X_1 + k_2X_2 + \cdots k_sX_s$
                \begin{equation}
                    \begin{aligned}
                        \eta &= k_1\eta_1 + k_2\eta_2 = k_1(a_1\alpha_1 + a_2\alpha_2 + \cdots + a_n\alpha_n) + k_2(b_1\alpha_1 + b_2\alpha_2 + \cdots + b_n\alpha_n)\\
                        &=(k_1a_1 + k_2b_1)\alpha_1 + (k_1a_2 + k_2b_2)\alpha_2 + \cdots + (k_1a_n + k_2b_n)\alpha_n
                    \end{aligned}
                \end{equation}
                $\Rightarrow X = k_1X_1 + k_2X_2$
            \end{enumerate}
        \end{theorem}
    \end{enumerate}

\end{property}

    % 第二章
    % \input{contents/section2.tex}

\end{document}

